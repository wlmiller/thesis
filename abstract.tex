This dissertation presents studies on self-organization in soft matter systems.
A wide variety of systems is studied, with the goal of understanding both the nonequilibrium and the equilibrium properties of this important process.

In Chapter~\ref{chap:janus}, we study the self-assembly of asymmetric Janus colloidal particles.
We identify and systematically describe the effect of the ratio of hydrophobic to hydrophilic surface area on the nonequilibrium processes and structure formation.

In Chapter~\ref{chap:aspherical}, we examine systems of hard, aspherical particles.
We demonstrate that the thermodynamics of self-organization of a system of these aspherical particle (either a system of identical particles or a polydisperse system of different-shaped particles) is well-predicted by a simple relationship between the crystallization pressure and two measures of particle asphericity borrowed from other fields.

In Chapter~\ref{chap:2dsoft}, we shift focus to systems of soft particles in two dimensions and on the surface of a sphere.
Soft particles are particles with a finite interaction potential at zero distance; such particles exhibit a surprisingly large variety of ordered structures at equilibrium.
A similar phenomenon is seen when the study is extended to soft particles on the surface of a sphere.

In Chapter~\ref{chap:brush}, we study the free energy of two-component polymer brush systems in which polymers of different length are patterned in alternating stripes of specified widths on the surface of a cylinder.
We present the dependence of the free energy on the polymer lengths and stripe width and a qualitative explanation of its functional form.

Finally, in Chapter~\ref{chap:pixel}, we approach the reverse self-assembly problem.
That is, we describe an algorithm for answering the reverse (and much more difficult) question, ``Given a specific desired target self-assembled structure, what interparticle interactions will yield a system which will self-assemble into that structure?''
We also describe a new model of interparticle interaction which should be able to generate interparticle interaction geometries with a high degree of flexibility.



%\section*{Hierarchical self-assembly of asymmetric Janus particles}
% From dumbbells to FCC crystals, we  study the self-assembly pathway
% of amphiphatic, spherical colloidal particles as a function of the size of the hydrophobic region using  molecular dynamics simulations.  Specifically, we analyze
% how local inter-particle interactions correlate to the final self-assembled aggregate and how they affect the dynamical pathway of structure formation.
% We present a detailed diagram separating the many phases that we find for different sizes of the hydrophobic area, and uncover a narrow region where particles self-assemble into hollow, faceted
% cages that could potentially find interesting engineering applications.
%
%\section*{Phase behavior of hard aspherical particles}
%We use numerical simulations to understand how random deviations from the ideal spherical shape affect the ability of hard particles to form
%fcc crystalline structures. Using a system of hard spheres as a reference,
%we determine the fluid-solid coexistence pressures of both shape-polydisperse and monodisperse systems of aspherical hard particles.
%We find that when particles are sufficiently isotropic,  the coexistence  pressure can be predicted from
%a linear relation involving the product of two simple geometric parameters characterizing the asphericity  of the particles.
%Finally, our results allow us to gain direct insight into the crystallizability limits of these systems by rationalizing
%empirical data obtained for analogous monodisperse systems~\cite{disorder1}.
%
%We use numerical simulations to study the crystallization of monodisperse systems of  hard aspherical particles.
%We find that particle shape and crystallizability can be easily related to each other when particles are
%characterized in terms of two simple and experimentally accessible order parameters: one based on the
%particle surface-to-volume ratio, and the other on the angular distribution of the
%perturbations away from the ideal spherical shape. We present a phase diagram obtained by exploring the
%crystallizability of  487 different particle  shapes across the two-order-parameter spectrum. Finally, we
%consider the physical properties of the crystalline structures accessible to aspherical particles, and
%discuss limits and relevance of our results.
%
%\section*{Exploiting Classical Nucleation Theory for reverse self-assembly}
%In this paper we introduce a new method to design interparticle interactions to target arbitrary crystal structures via the process of self-assembly.
%We show that it is possible   to exploit the curvature of the crystal nucleation free-energy barrier to sample and select optimal
%interparticle interactions for self-assembly into a desired structure.
%We apply this method to find interactions to target two simple crystal structures: a crystal with simple cubic symmetry
%and a two-dimensional plane with square symmetry embedded in a three-dimensional space.
%Finally, we discuss the potential and limits of our method and
%propose a general model by which a functionally infinite number of different interaction geometries may be constructed
%and to which our reverse self-assembly method could in principle be applied.
%
%\section*{Two-dimensional packing of soft particles and the soft generalized Thomson problem}
%We perform numerical simulations of \edit{a model of} purely repulsive
% soft colloidal particles interacting via a generalized elastic potential and
%constrained to a two-dimensional plane and to the surface of a spherical shell.
%For the planar case, we compute the phase diagram in terms of the system's rescaled density and temperature.
%We find that a large number of ordered phases becomes accessible at low temperatures as the density of the system
%increases, and we study systematically how structural variety depends on the functional shape
%of the pair potential.
%For the spherical case, we revisit the generalized Thomson problem for small numbers of particles $N\leq12$
%and identify, enumerate and compare the minimal energy polyhedra established by the location of the
%particles to those of the corresponding electrostatic system.
%
%\section*{Free energy of alternating two-component polymer brushes on cylindrical templates}
%We use computer simulations to investigate the stability of a two-component polymer brush de-mixing
%on a curved template into phases of different morphological properties. It has been previously shown via molecular dynamics simulations that immiscible chains having different length and anchored to a cylindrical template will phase separate into striped phases of different widths oriented perpendicularly to the cylindrical axis.
%We calculate free energy differences for a variety of stripe widths, and extract simple relationships between the sizes of the two polymers, $N_1$ and $N_2$, and the free energy dependence on the stripe width.
%We explain these relationships using simple physical arguments based upon previous theoretical work on the free energy of polymer brushes.
%
